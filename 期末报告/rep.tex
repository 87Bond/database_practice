\documentclass[a4paper,12pt]{article}

\PassOptionsToPackage{fontset=fandol}{ctex}

\usepackage{./SCUformat}
\usepackage{./cover}
\usepackage{booktabs}
\usepackage{longtable}

\begin{document}

\title{医院门诊挂号系统}
\author{郑仕博}
\adviser{段磊}
\college{计算机学院}
\major{计算机科学与技术}
\groupname{第四组}
\members{吴正博,高俊翔}
\date{\today}

\makecover

\tableofcontents
\clearpage

\section{引言}
\subsection{目的}
本系统的目的是围绕真实门诊挂号场景构建一套可运行的示例系统,覆盖号源查询、挂号、支付/取消、就诊记录与站内消息等完整业务链,并在实现过程中突出数据库的核心能力:关系建模、主外键约束与事务一致性控制。

\subsection{项目范围}
系统覆盖门诊挂号的核心业务链:账号与角色、科室/医生信息、号源(排班)管理、在线挂号、余额支付与取消退款、就诊病历、站内消息与权限控制。系统包含四类角色:患者、医生、科室管理员、系统管理员。

\subsection{需求分析}
医院挂号系统旨在解决传统挂号模式中存在的排队时间长、号源管理混乱、信息查询不便等问题,通过数字化方式实现挂号流程的规范化、高效化,同时满足患者、医生、医院管理员等不同角色的核心业务需求,为医院诊疗工作的有序开展提供技术支撑。本系统需保证数据的安全性、操作的易用性以及功能的完整性,符合医院日常运营的业务逻辑。

\subsection{文档概览}
第 1 部分说明背景与范围;第 2 部分给出系统概览与典型流程;第 3 部分描述整体架构与关键实现策略;第 4 部分给出数据库 E-R 图、关系模式与约束;第 5 部分说明各业务模块的实现;第 6 部分概述人机界面设计;第 7 部分介绍测试与验收;第 8 部分为需求矩阵;第 9 部分总结与展望;第 10 部分为附录,包含项目目录说明、环境要求、配置文件说明、部署与运行、示例数据说明及演示账号与建议流程。

\subsection{术语与缩略语}
E-R 图(实体联系图)、PK(主键)、FK(外键)、ACID(事务特性)、REST(接口风格)、Slot(号源/时间段)、Reg(挂号记录)、RBAC(基于角色的访问控制,本文实现为简化版)。

\section{系统概览}
\subsection{系统目标}
以“数据库为中心”完成一个可运行的 Web 项目,能够体现数据库原理课程的核心点:关系模式设计、主外键约束、典型增删改查、关键业务规则的一致性维护(容量、金额、状态流转)以及面向多角色的权限控制。

\subsection{系统定位与场景假设}
本系统定位于教学场景下的门诊挂号流程演示,围绕“号源、挂号、支付、消息与权限”等数据库课程的典型问题展开。为保持实验聚焦,支付方式采用账户余额扣款,不引入第三方支付;鉴权采用 \texttt{token=userId} 的简化方案;数据库建表与数据初始化通过脚本手动执行,突出关系模式与约束设计的可复现性。演示数据主要由 \texttt{sql/sample\_data.sql} 提供,便于在有限时间内覆盖完整业务链路。

\subsection{角色与权限}
\begin{itemize}
  \item 患者(patient):注册/登录;查询科室、医生与可预约号源;创建挂号、支付/取消、充值;查看挂号与就诊记录;在满足规则下发送站内消息。
  \item 医生(doctor):查看某日挂号患者列表;为已挂号记录录入病历;向患者/科室管理员/系统管理员发送消息(受“已支付挂号+10 天有效期”约束)。
  \item 科室管理员(dept\_manager):维护本科室医生归属;为医生创建号源(日期、起止时间、容量、费用、备注);查看本科室号源与挂号记录;按规则向本科室医生/患者发送消息。
  \item 系统管理员(admin):创建医生/科室管理员/系统管理员账号;冻结/解冻账号;通过消息机制发送系统通知。
\end{itemize}

\subsection{团队与分工}
\begin{table}[htbp]
\centering
\small
\begin{tabular}{p{2cm} p{5cm} p{3.5cm}}
\toprule
成员 & 核心职责 & 备注 \\
\midrule
郑仕博 & 前端美化和后端与核心业务(挂号、支付、取消、权限) & 接口联调与验收准备 \\
吴正博 & 前端页面与交互、消息与工作台体验 & 交互打磨与展示支持 \\
高俊翔 & 数据库建模、SQL 脚本、ER 图与文档 & 演示录屏与流程串联 \\
\bottomrule
\end{tabular}
\end{table}

\subsection{典型业务流程}
\begin{enumerate}
  \item 管理员初始化:执行 \texttt{sql/schema.sql} 建表,执行 \texttt{sql/sample\_data.sql} 插入样例数据;使用系统管理员账号登录进行账号维护,使用科室管理员账号登录进行号源维护。
  \item 患者挂号:选择科室与日期查询医生号源(含起止时间、余量与费用)$\rightarrow$ 选择 \texttt{slotId} 创建挂号 $\rightarrow$ 余额支付或取消(已支付按 90\% 退款)。
  \item 医生就诊:医生端按日期查看挂号患者列表 $\rightarrow$ 对某挂号记录录入病历(诊断、治疗方案、医嘱)$\rightarrow$ 患者在个人中心查看就诊记录。
  \item 消息沟通:基于“已支付挂号关系 + 双方最近消息 10 天内活跃 + 角色限制”校验发送权限;支持会话列表与会话明细查询。
\end{enumerate}

\section{系统架构}
\subsection{总体架构}
系统采用前后端分离三层结构:
\begin{itemize}
  \item 前端:Vue2,实现登录/注册、挂号页、消息页、个人中心与各角色工作台;通过代理将 \texttt{/api} 请求转发到后端。
  \item 后端:Spring Boot + MyBatis,提供 REST 接口(统一前缀 \texttt{/api}),业务逻辑集中在 Service 层,DAO 使用 Mapper 访问 PostgreSQL。
  \item 数据库:PostgreSQL,采用“手动执行脚本建库建表”的方式,保证验收时数据库模式清晰可复现。
\end{itemize}

\subsection{技术选型与原因}
\begin{itemize}
  \item Spring Boot:快速搭建 REST 服务,便于按模块组织 Controller/Service/Mapper。
  \item MyBatis:SQL 可控、便于体现课程中对查询与事务的设计能力。
  \item PostgreSQL:支持标准 SQL 与完整约束语义,适合课程要求的关系模型实现。
  \item Vue2:开发效率高,便于按角色组织页面与交互流程。
\end{itemize}

\subsection{接口规范与前后端协作}
\begin{itemize}
  \item 接口统一前缀为 \texttt{/api},前后端以 JSON 进行数据交互。
  \item 登录成功后返回 \texttt{userId, username, role, token},其中 \texttt{token} 简化为 \texttt{userId}。
  \item 前端将 \texttt{token} 存入本地缓存,并在请求头携带 \texttt{X-User-Id} 供后端识别用户与角色。
  \item 开发环境下前端通过代理将 \texttt{/api} 请求转发至后端服务,避免跨域问题。
\end{itemize}

\subsection{鉴权与会话策略(实验简化)}
为了聚焦数据库与业务一致性,本项目未引入完整 JWT/Spring Security。登录成功后返回的 \texttt{token} 简化为 \texttt{userId},前端在请求头携带 \texttt{X-User-Id},后端据此识别当前用户并做角色校验。

\subsection{接口调用与数据流}
系统数据流按“前端页面 $\rightarrow$ REST 接口 $\rightarrow$ Service 业务逻辑 $\rightarrow$ Mapper 持久化 $\rightarrow$ PostgreSQL”的路径执行。前端负责参数收集与状态展示,后端负责业务校验与事务一致性,数据库负责约束与持久化。数据库脚本与示例数据独立于应用运行,可在验收时直接复现完整环境。

\subsection{关键一致性策略}
\begin{itemize}
  \item 号源容量一致性:创建/取消挂号在同一事务内同时写入 \texttt{registration} 并更新 \texttt{doctor\_time\_slot.booked\_count},避免出现“挂号成功但容量未扣减”等不一致。
  \item 事务边界:挂号创建、支付与取消均以事务方式执行,保证“状态更新 + 金额/余量变更”原子提交或回滚。
  \item 金额统一以“分”存储:\texttt{patient.money}、\texttt{doctor\_time\_slot.fee}、\texttt{registration.reg\_fee} 统一使用整数分,避免浮点误差;前端录入“元”,后端转换为“分”落库。
  \item 业务约束校验:不可预约过去日期;号源不足禁止挂号;同一患者同一号源仅允许一条未取消记录;号源时间段不可重叠;取消已支付挂号按 90\% 退款;消息需满足“付费挂号关系 + 10 天有效期”。
\end{itemize}

\section{数据设计}
\subsection{ER 图}
\begin{figure}[H]
  \centering
  \includegraphics[width=0.95\textwidth]{../requirements/ER图.png}
  \caption{数据库核心实体关系图}
\end{figure}

\subsection{关系模式图}
\begin{figure}[H]
  \centering
  \includegraphics[width=0.98\textwidth]{../requirements/数据库关系图.png}
  \caption{数据库关系模式图}
\end{figure}

\subsection{字段设计说明}
\begin{itemize}
  \item 统一用户表 + 角色子表:\texttt{user\_directory} 负责基础身份信息,\texttt{patient}/\texttt{doctor}/\texttt{department\_manager}/\texttt{"system\_user"} 存放角色扩展字段。
  \item 金额字段统一使用“分”:\texttt{patient.money}、\texttt{doctor\_time\_slot.fee}、\texttt{registration.reg\_fee} 避免浮点误差。
  \item 时间段双存储:\texttt{start\_time}/\texttt{end\_time} 供约束校验,\texttt{time\_slot} 与 \texttt{reg\_time\_slot} 作为可读展示。
  \item 状态字段枚举化:\texttt{reg\_status} 与 \texttt{pay\_status} 用于控制流程与退款逻辑。
\end{itemize}

\subsection{设计要点与规范}
\begin{itemize}
  \item 角色扩展的可维护性:统一用户表避免重复存储基础信息,角色表只保存角色特有字段,便于权限校验与扩展。
  \item 号源与挂号的分离:号源负责容量与时间段,挂号记录负责状态流转,两者通过 \texttt{slot\_id} 关联,利于统计与一致性维护。
  \item 消息的可追溯性:消息记录保留发送方/接收方与时间戳,便于验收解释沟通路径。
  \item 命名与单位规范:字段命名保持语义一致,金额使用“分”,时间采用“日期 + 起止时间”的组合表达。
\end{itemize}

\subsection{关系模式与数据字典}
表结构与字段含义以 \texttt{sql/schema.sql} 和 \texttt{requirements/数据表设计.md} 为准,核心表如下:

\begin{longtable}{p{3cm} p{8cm} p{3cm}}
\toprule
表名 & 核心字段(节选) & 说明 \\
\midrule
\endhead
\texttt{user\_directory} & \texttt{user\_id, user\_name, phone, password, create\_time, is\_active} & 统一用户表(四类角色共享) \\
\texttt{patient} & \texttt{user\_id, gender, age, id\_card, money} & 患者扩展信息,\texttt{money} 单位分 \\
\texttt{department} & \texttt{department\_id, department\_name, introduction, location} & 科室主数据 \\
\texttt{doctor} & \texttt{user\_id, doctor\_name, department\_id, title, specialty} & 医生信息,关联科室 \\
\texttt{department\_manager} & \texttt{user\_id, department\_id, work\_phone} & 科室管理员,关联科室 \\
\texttt{"system\_user"} & \texttt{user\_id, work\_phone} & 系统管理员 \\
\texttt{doctor\_time\_slot} & \texttt{slot\_id, doctor\_id, slot\_date, start\_time, end\_time, capacity, booked\_count, fee} & 医生号源(排班) \\
\texttt{registration} & \texttt{reg\_id, patient\_id, doctor\_id, reg\_date, reg\_time\_slot, reg\_status, reg\_fee, pay\_status, slot\_id} & 挂号记录与状态流转 \\
\texttt{medical\_record} & \texttt{record\_id, reg\_id, diagnosis, treatment\_plan, advice} & 病历,一次挂号对应一条记录 \\
\texttt{message} & \texttt{message\_id, title, content, create\_user\_id, target\_user\_id, create\_time, is\_valid} & 站内消息 \\
\bottomrule
\end{longtable}

\subsection{关键约束与规则落地}
\begin{itemize}
  \item 主外键约束:在 \texttt{sql/schema.sql} 中通过 FK 约束保障实体引用完整性(如 \texttt{registration.patient\_id} 引用 \texttt{patient.user\_id})。
  \item 手机号唯一:\texttt{user\_directory.phone} 设置唯一约束,避免重复注册。
  \item 号源时间段不重叠:科室管理员创建号源时,后端在同一医生同一日期内检查 \texttt{start\_time/end\_time} 区间是否与已有号源重叠。
  \item 容量控制:创建挂号前校验 \texttt{capacity > booked\_count};成功后在事务内将 \texttt{booked\_count + 1},取消时 \texttt{booked\_count - 1}。
  \item 同号源重复挂号限制:创建挂号前按 \texttt{(patient\_id, slot\_id)} 查询是否存在未取消记录,避免重复占号。
  \item 取消退款:已支付挂号取消按 90\% 退款并回写余额;未支付直接取消。
  \item 消息权限与时效:发送消息需满足“已支付挂号关系”且双方最近消息在 10 天内;超期仅可查看历史,不允许继续发送。
\end{itemize}

\subsection{一致性控制示例}
\begin{enumerate}
  \item 挂号创建:在同一事务内写入 \texttt{registration} 并更新号源余量,防止并发下出现“超卖”。
  \item 支付与取消:支付记录 \texttt{paid\_at} 并更新 \texttt{pay\_status};取消时按规则退款并回写余额。
  \item 消息时效:发送消息前校验“已支付挂号关系 + 10 天活跃期”,不满足条件则拒绝创建消息记录。
\end{enumerate}

\subsection{状态字段约定}
\begin{itemize}
  \item 挂号状态 \texttt{registration.reg\_status}:\texttt{Booked}(已预约)、\texttt{Canceled}(已取消)、\texttt{Finished}(已完成,就诊流程可扩展)。
  \item 支付状态 \texttt{registration.pay\_status}:\texttt{Unpaid}(未支付)、\texttt{Paid}(已支付)、\texttt{Refunded}(已退款/作废)。
\end{itemize}

\subsection{脚本与示例数据}
项目不自动建表。验收或演示时,建议使用数据库管理工具连接 PostgreSQL 并执行:\texttt{sql/schema.sql} 创建表结构,\texttt{sql/sample\_data.sql} 插入示例数据。示例数据覆盖:
\begin{itemize}
  \item 10 个科室(\texttt{D001}~\texttt{D010})
  \item 10 个患者、10 个医生、10 个科室管理员、10 个系统管理员
  \item 多个医生排班号源与多条挂号记录(包含已支付、未支付与取消状态)
  \item 初始病历、站内消息与操作日志等辅助数据
\end{itemize}

\section{组件设计}
\subsection{账号与用户管理}
\begin{itemize}
  \item 患者注册:\texttt{POST /api/auth/patient/register},写入 \texttt{user\_directory} 与 \texttt{patient}(统一用户 + 角色扩展信息)。
  \item 登录:\texttt{POST /api/auth/login},支持使用 \texttt{userId} 或手机号登录;返回 \texttt{role} 与 \texttt{token=userId}。
  \item 修改密码:\texttt{POST /api/auth/change-password}。
  \item 账号创建与冻结/解冻:系统管理员通过 \texttt{/api/users/*} 创建医生/科室管理员/系统管理员账号,并通过更新 \texttt{user\_directory.is\_active} 控制账号可用性。
\end{itemize}

\subsection{科室、医生与号源}
\begin{itemize}
  \item 科室列表:\texttt{GET /api/departments},提供科室名称与地点,供挂号页筛选。
  \item 医生列表:\texttt{GET /api/doctors?departmentId=\&date=},按科室与日期返回医生信息与可预约号源(含起止时间、余量与费用)。
  \item 号源查询:\texttt{GET /api/registrations/slots} 支持按科室/日期检索号源。
  \item 号源创建:科室管理员通过 \texttt{POST /api/dept/slots} 为本科室医生创建号源;时间段校验 \texttt{start\_time < end\_time} 且不重叠。
\end{itemize}

\subsubsection{号源创建(科室管理员)实现步骤}
\begin{enumerate}
  \item 权限校验:当前用户必须是科室管理员,且只能为本科室医生创建号源。
  \item 参数校验:日期、起止时间、容量、费用必填,且满足 \texttt{start\_time < end\_time}、\texttt{capacity > 0}。
  \item 重叠检查:查询该医生该日期的已有号源,若存在区间重叠(非 \texttt{end <= existing.start} 且非 \texttt{start >= existing.end})则拒绝创建。
  \item 落库:生成 \texttt{slot\_id};将展示字段 \texttt{time\_slot} 固化为 “HH:mm-HH:mm”;费用由“元”换算为“分”;初始化 \texttt{booked\_count=0} 与 \texttt{status=OPEN}。
\end{enumerate}

\subsection{挂号、支付与取消}
\begin{itemize}
  \item 创建挂号:\texttt{POST /api/registrations},必须选择 \texttt{slotId};校验日期合法、医生科室匹配、号源余量;写入 \texttt{registration} 并扣减余量。
  \item 支付:\texttt{POST /api/registrations/\{regId\}/pay},从 \texttt{patient.money} 扣减 \texttt{registration.reg\_fee},写入支付时间并更新状态。
  \item 取消:\texttt{POST /api/registrations/\{regId\}/cancel},更新挂号状态;已支付按 90\% 退款;释放号源余量。
  \item 查询挂号:患者 \texttt{GET /api/registrations},医生端 \texttt{GET /api/doctor/registrations}。
\end{itemize}

\subsection{账户余额与充值}
\begin{itemize}
  \item 查询余额:\texttt{GET /api/patient/me} 返回当前账户余额与基本信息。
  \item 充值:\texttt{POST /api/patient/recharge} 以“元”为输入单位,后端转换为“分”入账。
  \item 支付校验:挂号支付时优先校验余额充足,不足则提示先充值。
\end{itemize}

\subsubsection{创建挂号实现步骤}
\begin{enumerate}
  \item 校验 \texttt{regDate} 不能早于当天。
  \item 校验医生属于所选科室,防止“科室/医生不匹配”造成脏数据。
  \item 通过 \texttt{slotId} 定位号源,并校验号源的医生/科室/日期与请求一致。
  \item 校验余量:\texttt{capacity > booked\_count} 才允许创建。
  \item 校验重复挂号:同一患者同一 \texttt{slotId} 若存在未取消记录则拒绝创建。
  \item 写入 \texttt{registration}:复制费用到 \texttt{reg\_fee};将 \texttt{reg\_time\_slot} 统一保存为 “HH:mm-HH:mm”;初始化 \texttt{reg\_status=Booked}、\texttt{pay\_status=Unpaid}。
  \item 更新号源:\texttt{booked\_count + 1}(与挂号写入处于同一事务内)。
\end{enumerate}

\subsubsection{支付与取消实现步骤}
\begin{enumerate}
  \item 支付:校验余额充足后扣减 \texttt{patient.money},并将 \texttt{pay\_status} 更新为 \texttt{Paid},记录 \texttt{paid\_at}。
  \item 取消:仅允许取消 \texttt{Booked} 状态;已支付按 90\% 退款并更新余额;将挂号状态置为 \texttt{Canceled/Refunded} 并释放号源(\texttt{booked\_count - 1})。
\end{enumerate}

\subsection{病历与就诊记录}
医生对挂号记录录入病历(诊断、治疗方案、医嘱),患者与医生端分别提供查询接口,保证一次挂号对应一份可追溯的就诊记录。接口包括 \texttt{POST /api/medical-records}(医生录入)、\texttt{GET /api/medical-records}(患者查询)与 \texttt{GET /api/medical-records/doctor}(医生查询)。

\subsection{站内消息}
消息模块支持收件箱/发件箱、会话列表(最近一条)与会话明细。发送消息时进行角色与关系校验,并结合“10 天活跃期”限制控制继续对话的权限。主要接口包括 \texttt{POST /api/messages}、\texttt{GET /api/messages}、\texttt{GET /api/messages/conversations} 与 \texttt{GET /api/messages/conversation}。

\subsubsection{消息权限与 10 天有效期}
\begin{enumerate}
  \item 角色限制:按发送方/接收方角色确定可通信范围(患者仅能向就诊医生或对应科室管理员发送;医生可向就诊患者/本科室管理员/系统管理员发送;科室管理员可向本科室医生与本科室就诊患者发送;系统管理员可向任意用户发送)。
  \item 关系校验:对“患者$\leftrightarrow$医生/科室管理员”等关系,要求存在近 10 天内的\textbf{已支付挂号}记录。
  \item 会话活跃校验:若双方最近一条消息时间超过 10 天,禁止继续发送(历史可查看)。
  \item 撤回/作废:通过 \texttt{message.is\_valid} 进行逻辑失效,不物理删除,便于追溯。
\end{enumerate}

\subsection{会话展示与作废逻辑}
\begin{itemize}
  \item 会话列表接口返回每位联系人最近一条消息,按时间倒序展示,便于快速定位活跃对话。
  \item 会话明细接口返回双方历史消息列表,保留完整时间线,便于复盘就诊沟通。
  \item 作废消息仅在逻辑层隐藏,不物理删除,满足课程对可追溯性的要求。
\end{itemize}

\section{人机界面设计}
前端按角色展示不同导航与工作台:
\begin{itemize}
  \item 登录/注册:患者注册与四类用户登录入口。
  \item 挂号页(患者):选择科室与日期,展示医生与可约号源,完成挂号创建。
  \item 消息页(全角色):会话列表 + 聊天明细 + 发送消息。
  \item 个人中心(全角色):查看个人信息、修改密码;患者可充值与管理挂号记录;医生可查看挂号与病历。
  \item 医生端:按日期查看患者挂号并录入病历。
  \item 科室端:维护医生归属、创建号源、查看本科室挂号记录。
  \item 系统管理端:创建账号、冻结/解冻、发送系统通知。
\end{itemize}

\subsection{页面与接口映射(节选)}
\begin{itemize}
  \item 登录/注册页:\texttt{/api/auth/patient/register}、\texttt{/api/auth/login}、\texttt{/api/auth/change-password}。
  \item 挂号页:\texttt{/api/departments}、\texttt{/api/doctors}、\texttt{/api/registrations}。
  \item 消息页:\texttt{/api/messages/conversations}、\texttt{/api/messages/conversation}、\texttt{/api/messages}。
  \item 个人中心:\texttt{/api/registrations}、\texttt{/api/medical-records}、\texttt{/api/patient/me}、\texttt{/api/patient/recharge}。
  \item 医生端:\texttt{/api/doctor/registrations}、\texttt{/api/medical-records}。
  \item 科室端:\texttt{/api/dept/doctors}、\texttt{/api/dept/slots}、\texttt{/api/dept/registrations}。
  \item 系统管理端:\texttt{/api/users/*}(创建账号、冻结/解冻)。
\end{itemize}

\subsection{角色工作台说明}
\begin{itemize}
  \item 患者端:围绕“挂号、支付、取消、消息、就诊记录”组织入口,主路径集中在挂号页与个人中心。
  \item 医生端:关注“当日挂号列表 + 病历录入 + 消息沟通”,以效率为导向减少跳转。
  \item 科室管理员端:以“号源创建/查看、医生归属维护、挂号统计”为主,支持快速管理号源。
  \item 系统管理员端:集中处理账号创建与冻结/解冻,并通过消息机制发布通知。
\end{itemize}

\subsection{交互流程示意}
\begin{enumerate}
  \item 患者端:登录/注册 $\rightarrow$ 选择科室与日期 $\rightarrow$ 查看医生与号源 $\rightarrow$ 创建挂号 $\rightarrow$ 支付或取消 $\rightarrow$ 查看就诊记录与消息。
  \item 医生端:登录 $\rightarrow$ 查看当天挂号患者 $\rightarrow$ 录入病历 $\rightarrow$ 发出就诊提醒或随访消息。
  \item 管理端:登录 $\rightarrow$ 创建账号/维护号源 $\rightarrow$ 查看挂号记录 $\rightarrow$ 发送系统通知。
\end{enumerate}

\section{测试与验收}
\subsection{测试数据准备}
\begin{enumerate}
  \item 执行 \texttt{sql/schema.sql} 创建表结构。
  \item 执行 \texttt{sql/sample\_data.sql} 插入示例数据,包含多角色账号与典型挂号记录。
  \item 使用系统管理员账号 \texttt{AD0001 / admin123} 登录,检查账号创建与冻结功能。
\end{enumerate}

\subsection{典型流程验证}
\begin{enumerate}
  \item 患者流程:注册/登录 $\rightarrow$ 查询号源 $\rightarrow$ 创建挂号 $\rightarrow$ 支付/取消 $\rightarrow$ 查看挂号与就诊记录。
  \item 医生流程:查看当日挂号列表 $\rightarrow$ 录入病历 $\rightarrow$ 向患者发送就诊提醒。
  \item 管理员流程:创建账号与号源 $\rightarrow$ 查看挂号记录 $\rightarrow$ 发送系统通知。
\end{enumerate}

\subsection{异常与约束场景}
\begin{itemize}
  \item 号源容量不足或日期非法时禁止挂号,提示用户调整时间段。
  \item 同一患者同一号源重复挂号被拒绝,避免重复占号。
  \item 余额不足时支付失败,需要先充值再支付。
  \item 消息超过 10 天未活跃时禁止继续发送,仅可查看历史记录。
\end{itemize}

\section{需求矩阵}
\begin{table}[H]
  \centering
  \begin{tabular}{p{4cm} p{9cm}}
    \toprule
    \textbf{需求} & \textbf{实现(主要接口/数据表/页面)} \\
    \midrule
    登录/注册与修改密码 & \texttt{/api/auth/*};\texttt{user\_directory}、\texttt{patient};登录注册页、个人中心 \\
    科室与医生号源查询 & \texttt{/api/departments}、\texttt{/api/doctors};\texttt{department}、\texttt{doctor}、\texttt{doctor\_time\_slot};挂号页 \\
    创建挂号、支付与取消 & \texttt{/api/registrations}、\texttt{/api/registrations/\{id\}/pay}、\texttt{/api/registrations/\{id\}/cancel};\texttt{registration}、\texttt{doctor\_time\_slot}、\texttt{patient};个人中心 \\
    号源维护(科室管理员) & \texttt{/api/dept/slots}、\texttt{/api/dept/doctors};\texttt{doctor\_time\_slot};科室端 \\
    病历与就诊记录 & \texttt{/api/medical-records};\texttt{medical\_record}、\texttt{registration};医生端、个人中心 \\
    站内消息与权限时效 & \texttt{/api/messages*};\texttt{message}、\texttt{registration};消息页 \\
    账号创建与冻结/解冻 & \texttt{/api/users/*};\texttt{user\_directory} 与角色子表;系统管理端 \\
    \bottomrule
  \end{tabular}
  \caption{需求—实现矩阵(节选)}
\end{table}

\section{总结与展望}
本项目以门诊挂号为核心场景,完成了从需求分析、数据库建模、接口实现到前端交互的完整闭环,能够体现课程强调的关系模式设计、约束表达与事务一致性控制。通过统一用户表、号源-挂号分离、消息与权限规则等设计,保证了多角色协同下的数据一致与可追溯性。

后续可扩展方向包括:
\begin{itemize}
  \item 引入更完整的鉴权与权限体系(JWT/RBAC),提升安全性与可维护性。
  \item 增加统计报表与运营类查询,丰富数据分析能力。
  \item 引入容器化部署与自动化测试,提升项目可交付性与稳定性。
\end{itemize}

\section{APPENDICES}
\subsection{项目目录说明}
以项目根目录为基准,主要内容如下:
\begin{itemize}
  \item \texttt{backend/}:后端 Spring Boot + MyBatis 源码与配置。
  \item \texttt{medical-regist/}:前端 Vue2 工程与页面资源。
  \item \texttt{sql/}:数据库建表与示例数据脚本(\texttt{schema.sql}、\texttt{sample\_data.sql})。
  \item \texttt{requirements/}:需求分析、数据表设计、模块接口等文档。
  \item \texttt{ppt/}:课程汇报 PPT 源文件。
  \item \texttt{期末报告/}:期末报告 LaTeX 源文件与相关资源。
  \item \texttt{README.md}、\texttt{SYSTEM\_README.md}:项目说明与运行指南。
  \item \texttt{演示视频.mp4}:功能演示录屏(位于项目根目录)。
\end{itemize}

\subsection{环境要求}
\begin{itemize}
  \item JDK 17
  \item Maven 3.9+
  \item Node.js 16+ 与 npm/pnpm
  \item PostgreSQL 16(本地默认 \texttt{localhost:5432})
\end{itemize}

\subsection{配置文件说明}
\begin{itemize}
  \item 后端数据库连接:\texttt{backend/backend/src/main/resources/application.properties}
  \item 前端开发代理:\texttt{medical-regist/vue.config.js}
  \item 前端请求封装:\texttt{medical-regist/src/api.js}
\end{itemize}

\subsection{部署与运行}
\begin{enumerate}
  \item 准备 PostgreSQL(本地 \texttt{localhost:5432}),连接后执行 \texttt{sql/schema.sql}(可选执行 \texttt{sql/sample\_data.sql})。
  \item 后端配置运行 mvn compile 和 mvn package 进行编译后,用 java -jar 运行生成的 jar 包。
  \item 前端安装依赖 npm install ,编译静态文件 npm run build ,配置服务器能够访问静态文件。
  \item 使用nginx将前端 /api/ 请求代理到后端使静态文件(即前端)可以向后端发送请求。
  \item 可以通过http://database.zhengshibo.top:2000/访问前端页面。
\end{enumerate}

\subsection{示例数据说明}
\begin{itemize}
  \item \texttt{sql/sample\_data.sql} 提供角色账号、科室、号源与挂号记录,覆盖演示所需的完整流程。
  \item 若不使用示例数据,患者仍可通过前端注册创建账号,并由管理员补充医生与号源。
\end{itemize}

\subsection{演示账号与建议流程}
若执行了 \texttt{sql/sample\_data.sql},可使用系统管理员账号 \texttt{AD0001 / admin123} 登录进行演示;患者可通过前端注册创建。建议演示顺序:患者挂号(创建/支付/取消)$\rightarrow$ 医生查看挂号并写病历 $\rightarrow$ 双方消息沟通 $\rightarrow$ 管理端创建账号与冻结解冻。

\end{document}
