%% 医院挂号系统课程设计汇报
%% !XeLaTeX & Biber
\PassOptionsToPackage{fontset=fandol}{ctex}
\documentclass[hyperref,UTF8,11pt,CJK,handout]{beamer}

\usetheme[chinese,MathFont=LM,BlockDisplay=colorful,CodeDisplay=listing,ContentMuticols=true,Background=true]{scu}

\usepackage{transparent}
\usepackage[autoplay]{animate}
\usepackage{tikz}
\usetikzlibrary{shadings,arrows,calc,decorations.pathmorphing,patterns}
\usepackage{array}
\usepackage{algorithm,algorithmic}
\usepackage{amsmath,amsfonts,amssymb}
\usepackage{mathtools}
\usepackage[english]{babel}
\usepackage{color}
\usepackage{xcolor}
\usepackage{url}
\usepackage{multicol,multirow}
\usepackage{ulem}
\usepackage{booktabs}
\usepackage{epstopdf,epsfig}
\usepackage{cprotect}
\usepackage{makecell}
\usepackage{listings}
\usepackage{subcaption}
\usepackage{varioref,cleveref,tcolorbox,biblatex}
\usepackage{graphicx}

\graphicspath{{image/}{resources/}}

\newcommand{\fverb}[1]{\texttt{#1}}

\title[医院挂号系统课程设计汇报]{\zihao{3} 医院挂号系统课程设计汇报}
\subtitle{数据库原理实验 · 终期验收}
\author[郑仕博 · 吴正博 · 高俊翔]{\noindent 郑仕博 \quad 吴正博 \quad 高俊翔}
\institute[四川大学计算机学院]{\noindent 四川大学 · 计算机学院\\数据库原理课程设计}

\begin{document}

\begin{frame}{目录}
	\inserttableofcontent
\end{frame}

\section{项目背景与目标}

\begin{frame}{课题选择与系统定位}
	\begin{itemize}
		\item 选题:\textbf{医院门诊挂号系统}——覆盖“挂号、就诊、消息通知、管理”完整业务链。
		\item 设计目标:
		\begin{itemize}
			\item 贴近真实医院场景,体现数据库在多角色协同中的作用;
			\item 用清晰的关系模式表达号源、挂号、就诊记录、消息等核心实体;
			\item 突出事务一致性与权限控制,便于老师从数据库角度评价。
		\end{itemize}
		\item 本组成员:郑仕博、吴正博、高俊翔(三人小组,满足课程要求)。
	\end{itemize}
\end{frame}

\begin{frame}{开发过程概览}
	\begin{itemize}
		\item 6--7 周:完成需求分析,明确四类角色与主要业务流程。
		\item 8--9 周:绘制 ER 图、整理数据表设计与接口列表,确定数据库模式。
		\item 10--14 周:后端接口与前端页面开发、联调测试,准备示例数据与日志。
		\item 15 周:整理代码与 SQL 脚本,制作本次汇报 PPT,并录制核心功能演示视频。
	\end{itemize}
\end{frame}

\section{项目简介}

\begin{frame}{项目概览(第 15 周验收版)}
	\begin{itemize}
		\item 项目名称:医院门诊挂号系统(数据库原理实验大作业)
		\item 系统目标:完成从账号、号源查询、在线挂号、支付/取消、病历、消息到权限控制的闭环。
		\item 技术栈:
		\begin{itemize}
			\item 后端:Spring Boot + MyBatis(REST:\texttt{/api})
			\item 前端:Vue2(按角色展示不同工作台)
			\item 数据库:PostgreSQL(\texttt{sql/schema.sql} 建表,\texttt{sql/sample\_data.sql} 初始化)
		\end{itemize}
	\end{itemize}
\end{frame}
\begin{frame}{团队与分工}
\begin{table}
\centering
\small
\begin{tabular}{p{2cm} p{5cm} p{3.5cm}}
\toprule
成员 & 核心职责 & 备注 \\
\midrule
郑仕博 & 后端与核心业务(挂号、支付、取消、权限) & 接口联调与验收准备 \\
吴正博 & 前端页面与交互、消息与工作台体验 & 演示录屏与流程串联 \\
高俊翔 & 数据库建模、SQL 脚本、ER 图与文档 & 需求与约束整理 \\
\bottomrule
\end{tabular}
\end{table}
\end{frame}


\section{系统实现}

\begin{frame}{角色与核心功能}
	\begin{itemize}
		\item 患者:查科室/医生/号源 \(\rightarrow\) 创建挂号 \(\rightarrow\) 支付/取消 \(\rightarrow\) 查看挂号与就诊记录 \(\rightarrow\) 站内信沟通
		\item 医生:查看当日挂号患者列表、录入病历、与患者/管理员消息互通
		\item 科室管理员:维护医生归属、创建号源、查看本科室挂号记录
		\item 系统管理员:创建账号、冻结/解冻、发送系统通知
	\end{itemize}
	\medskip
	\textbf{鉴权简化}
	\begin{itemize}
		\item 登录后返回 \texttt{token=userId},前端请求头携带 \texttt{X-User-Id}(满足课程作业范围)。
	\end{itemize}
\end{frame}

\begin{frame}{核心业务链路(演示重点)}
	\begin{columns}[t]
		\column{0.48\textwidth}
		\textbf{挂号与支付}
		\begin{itemize}
			\item 号源:\texttt{doctor\_time\_slot}(日期 + 起止时间 + 容量 + 费用)
			\item 创建:\texttt{POST /api/registrations}(必须传 \texttt{slotId})
			\item 支付:余额扣款,记录 \texttt{paidAt},状态 \texttt{Unpaid}\(\rightarrow\)\texttt{Paid}
			\item 取消:患者 90\% 退款;释放号源 \texttt{booked\_count}
		\end{itemize}
		\column{0.48\textwidth}
		\textbf{消息与协同}
		\begin{itemize}
			\item 前置关系:必须存在已支付挂号(患者 \(\leftrightarrow\) 医生/科室管理员)
			\item 时效限制:双方最近一条消息超过 10 天,禁止继续发送(历史可查)
			\item 会话:\texttt{/api/messages/conversations}(最近一条)+\ \texttt{/api/messages/conversation}(全量)
		\end{itemize}
	\end{columns}
\end{frame}

\section{数据库设计}

\begin{frame}{ER 结构概览}
	\begin{columns}
		\column{0.58\textwidth}
		\begin{figure}
			\centering
			\includegraphics[width=\linewidth]{db-er.png}
			\caption{核心实体关系图(节选)}
		\end{figure}
		\column{0.38\textwidth}
		\begin{itemize}
			\item 统一用户表 \texttt{user\_directory} 与四类角色子表。
			\item 号源、挂号、病历串联核心业务流程。
			\item \texttt{message} 和 \texttt{audit\_log} 辅助沟通、追踪操作。
			\item 所有外键与约束集中在 \texttt{sql/schema.sql},便于评审直接查看。
		\end{itemize}
	\end{columns}
\end{frame}

\begin{frame}{关键约束与数据策略}
	\begin{itemize}
		\item \textbf{号源管理}
		\begin{itemize}
			\item 同一医生同一天的时间段不得重叠;\texttt{start\_time < end\_time},容量必须大于零。
			\item 通过唯一索引 \texttt{(patient\_id, slot\_id)} 保证同一号源只存在一条未取消记录。
		\end{itemize}
		\item \textbf{挂号与支付}
		\begin{itemize}
			\item \texttt{reg\_fee} 与 \texttt{doctor\_time\_slot.fee} 保持一致,单位均为分,避免精度误差。
			\item 取消逻辑更新 \texttt{booked\_count} 并追加审计日志,确保数据一致性。
		\end{itemize}
		\item \textbf{消息与会话}
		\begin{itemize}
			\item 数据库层面存储最近消息时间,接口在发送时校验“10 天活跃”规则。
			\item 历史消息不会被删除,仅在 \texttt{is\_valid=false} 时隐藏,方便追溯。
		\end{itemize}
	\end{itemize}
\end{frame}

\section{演示与总结}

\begin{frame}{演示流程建议(8~10 分钟)}
	\begin{enumerate}
		\item \textbf{项目概览}:目标、技术栈、角色与权限。
		\item \textbf{患者流程}:查号源 → 挂号 → 支付/取消 → 消息/就诊记录。
		\item \textbf{医生流程}:查看当天挂号 → 写病历 → 发消息提醒。
		\item \textbf{管理流程}:创建号源/账号、冻结解冻或系统通知。
		\item \textbf{数据库说明}:ER 图、关键约束与一致性策略。
	\end{enumerate}
	\medskip
\end{frame}

\begin{frame}{总结与展望}
	\begin{itemize}
		\item 已完成:需求 \(\rightarrow\) 概要设计 \(\rightarrow\) 实现与联调 \(\rightarrow\) 第 15 周验收材料(PPT + 录屏 + 代码 + SQL)。
		\item 数据库设计重点:实体关系清晰、主外键与约束完整、关键业务规则可追溯。
		\item 后续扩展:真实 JWT/RBAC、统计报表、容器化与自动化测试。
	\end{itemize}
\end{frame}

\end{document}
